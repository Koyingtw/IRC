\documentclass[aspectratio=169]{beamer}
\usefonttheme{serif}
\usepackage{xeCJK}
\usepackage{fontspec}
\usepackage{graphicx}
\usepackage{listings}
\usepackage{xcolor}
\usepackage{indentfirst}
\usepackage{tikz}
\usepackage{amssymb}
\usepackage{amsthm}
\usepackage{amsmath}
\usepackage{tabularx}
\usepackage{hyperref}
\usepackage{ulem}
\usepackage{version}
\usepackage{thmtools}
\usepackage{qtree}
\usepackage{algpseudocode}
\usepackage{mathtools}
\usepackage{multicol}
\usepackage{xcolor}

\AtBeginDocument{%
    \DeclareSymbolFont{pureletters}{T1}{\mathfamilydefault}{\mddefault}{it}%
    }

\XeTeXlinebreaklocale "zh"
\XeTeXlinebreakskip = 0pt plus 1pt

\setCJKmainfont[AutoFakeBold,AutoFakeSlant]{SourceHanSansTW-Regular.otf}
\usetikzlibrary{arrows,decorations.markings,decorations.pathreplacing}
\newenvironment{Hint}{\noindent\textbf{Hint.}}{}

\tikzstyle {graph node} = [circle, draw, minimum width=1cm]
\tikzset{edge/.style = {decoration={markings,mark=at position 1 with %
            {\arrow[scale=2,>=stealth]{>}}},postaction={decorate}}}

\lstset{
    basicstyle=\ttfamily\normalsize,
    numberstyle=\normalsize,
    numbers=left,
    stepnumber=1,
    numbersep=3pt,
    commentstyle=\color{black!50},
    keywordstyle=\color{white!0!blue},
    stringstyle=\color{black!50!green},
    showspaces=false,
    showstringspaces=false,
    showtabs=false,
    tabsize=4,
    captionpos=b,
    breaklines=true,
    breakatwhitespace=false,
    escapeinside={\%*}{*)},
    morekeywords={*}
}

\AtBeginSection[]{
  \begin{frame}
  \vfill
  \centering
  \begin{beamercolorbox}[sep=8pt,center,shadow=true,rounded=true]{title}
    \usebeamerfont{title}\insertsectionhead\par%
  \end{beamercolorbox}
  \vfill
  \end{frame}
}

\title{Codeforces 799 (Div.4) 題解}
\author{Koying}
\date{2022-06-15}

\usetheme{Madrid}
\usecolortheme{default}
\setbeamertemplate{itemize items}[square]
\setbeamertemplate{enumerate items}[default]
\setbeamertemplate{blocks}[default]
\lstdefinestyle{myStyle}{
    belowcaptionskip=1\baselineskip,
    breaklines=true,
    frame=none,
    numbers=none, 
    basicstyle=\footnotesize\ttfamily,
    keywordstyle=\bfseries\color{green!40!black},
    commentstyle=\itshape\color{purple!40!black},
    identifierstyle=\color{blue},
    backgroundcolor=\color{gray!10!white},
}

\begin{document}

    \begin{frame}
        \titlepage
    \end{frame}     

    \begin{frame}{reference}
    
    	\begin{itemize}
    		\item 本次題解的程式碼都可以在 https://github.com/Koyingtw/IRC-lecture/20220615 找到 
    	\end{itemize}
    	
    \end{frame}
    
    \begin{frame}{Codeforces 799 A}
    	\begin{block}{CF 799A Marathon}
    		有三個人 a, b, c, d 在跑步,給他們目前跑的距離,求有幾人的距離比 a 遠
    	\end{block}

		\begin{itemize}
			\item 直接做就好
			\item 參考程式:PA.cpp
		\end{itemize}		    	
    \end{frame}
    
    \begin{frame}{Codeforces 799 B}
    	\begin{block}{CF 799B All Distinct}
    		Sho 得到一個長度為 $n$ 的序列,他會進行多次操作,每次操作會將序列中的兩個數字刪除,直到沒有任兩個相同的數字為止\\
    		求最後序列最長可以多長\\
    		$(t \le 10^3, n \le 50)$
    	\end{block}
    	
    	\begin{itemize}
    		\item 每種數字最多就只會剩下一個,因此最終目標是要讓剩下一個的數量最大化
    		\item 若有一個數字 $a$ 一開始有奇數個,那每次刪掉兩個,最後就會剩下一個
    		\item 若有兩個數字 $a,b$ 一開始都有偶數個,那我們只要先將 $a,b$ 都 $-1$,那就可以將這兩個轉為奇數
    		\item 最終答案就是一開始奇數的數量加上可以從偶數轉成奇數的數量
    		\item 參考程式:PB.cpp
    	\end{itemize}
    \end{frame}
    
    \begin{frame}{Codeforces 799 C}
    	\begin{block}{CF 799C Where's the Bishop?}
    		給定一西洋棋中的主教可以走到的格子(原先在的格子也算),求出原先的格子是哪一格
    	\end{block}
    	
    	\begin{itemize}
    		\item 可以發現到,兩條對角線會在原點交會,形成 X 狀
    		\item 若 $(x,y),(x+1,y+1),(x+1,y-1),(x-1,y-1),(x-1,y-1)$ 都可走的話,那麼 $x,y$ 就是原點
    		\item 參考程式:PC.cpp
    	\end{itemize}
    \end{frame}
    
    \begin{frame}{Codeforces 799 D}
    	\begin{block}{CF 799D The Clock}
    		有一個 24 小時制的時鐘,Victor 每 $x$ 分鐘會看一次時鐘,問 Victor 在同樣的時間看不超過一次的情況下,會看到幾個回文的時間
    	\end{block}
    	
    	\begin{itemize}
    		\item 可以證明由於看不超過 1 次,因此最多就只會看 $24 \times 60$ 次
    		\item 按照題目實作即可,時鐘進位的算法請參考下面的程式碼
    		\item 輸入可用 stringstream 來快速處理
    		\item Time complexity:\ $\mathcal{O}(24 \times 60)$
    		\item 參考程式:PD.cpp
    	\end{itemize}
		
    \end{frame}
    
    \begin{frame}
    
    	\center 時鐘運行範例
		\lstinputlisting[
    		language=C++,
    		style=myStyle,
    		frame=tb
    	]{PD_clock.cpp}

    	\center 回文判斷範例    	
		\lstinputlisting[
    		language=C++,
    		style=myStyle,
    		frame=tb
    	]{PD_isPalindron.cpp}

		
    \end{frame}
    
    \begin{frame}{Codeforces 799 E}
    	\begin{block}{CF 799D Binary Deque}
    		有一個長度為 $n$ 的 01 字串,你可以從頭或從尾移除數個元素,求最少移除幾個字元能使剩下的字元洽有 $s$ 個 1\\
    		$(n, s \le 2 \cdot 10^5)$
    	\end{block}
    	
		\begin{itemize}
			\item 可以觀察到,左界越往右則右界也只會越往右,具有單調性
			\item 枚舉左界或是右界,利用前綴和差分二分搜出右界或是左界
			\item Time complexity:\ $\mathcal{O}(n\log n)$
			\item 參考程式: PE.cpp
		\end{itemize}		    	
    \end{frame}
    
    \begin{frame}{Codeforces 799 F}
		\begin{block}{CF 799F 3SUM}
			給一個長度為 $n$ 的序列,求是否有可能任取三個元素使得其和的尾數為 $3$\\
			$(n \le 2 \cdot 10^5)$
		\end{block}		    
    
    		\begin{itemize}
    			\item $\mathcal{O}(n^3)$ 的作法顯然不可行
    			\item 可以觀察到,三個元素和的尾數是否是 $3$ 只跟三個元素的尾數和有關
    			\item 紀錄每個尾數的數量
    			\item 枚舉三個元素的尾數
    			\item Time complexity:\ $\mathcal{O}(nlogn)$
    			\item 參考程式: PF.cpp
    		\end{itemize}
    \end{frame}
    
    \begin{frame}{Codeforces 799 G}
    		\begin{block}{CF 799G 2\ \^\ Sort}
    			有一個數列 $a$,你可以任選一個長度為 $k+1$ 的區間 $[i,i+k]$,並進行一次操作,使得區間內的值變為 $2^0 \cdot a_i,\ 2^1 \cdot a_{i+1},\ ...,\ 2^k \cdot a_{i+k}$\\
    			求有幾種個區間經過操作之後其區間內的元素呈現嚴格遞增\\
    			$(n \le 2 \cdot 10^5)$
    		\end{block}
    		
    		\begin{itemize}
    			\item 若 $[a_i,\ a_{i+1}]$ 合法,那麼 $a_i < a_{i+1} \cdot 2$
    			\item 若 $[l,r]$ 之間都合法,那麼其中最多包含 $r-l+1-k$ 個長度為 $k+1$ 的區間
    			\item 找出所有合法的連續區間,即可找到答案
    			\item Time complexity:\ $\mathcal{O}(n)$
    			\item 參考程式: PG.cpp
    		\end{itemize}
    \end{frame}
    
    \begin{frame}{Codeforces 799 H}
    		\begin{block}{Codeforces 799H Gambling}
    			Marian 在賭場賭博,有一個遊戲每次會隨機跳一個數字出來,若猜對獎金加倍,猜錯則是獎金折半\\
    			Marian 能夠知道接下來每一局遊戲都會出現什麼數字 $x_i$,你可以指定一個數字 $a$ 使得 Marian 接下來都猜測 $a$,以及一個區間 $[l,r]$ 代表 Marian 會連續玩 $[l,r]$ 的遊戲\\
    			求 Marian 最多可以獲得多多獎金,他一開始有一塊錢\\
    			$(n \le 2 \cdot 10^5)$
    		\end{block}
    		
    		\begin{itemize}
    			\item 可以將題目所求轉換成贏得該局則 $+1$,輸則 $-1$,求最大獎金
    			\item 假設我們要猜的是 $a$,那麼對於每次遊戲出現的數字 $x_i$,若 $x_i=a$,則我們將該局結果 $r_{a,i}$ 設為 $+1$,反之設為 $-1$,那麼猜 $a$ 的最大獎金就是 $r_a$ 的最大區間和
    			\item 最大區間和算法可由 (當前前墜和 - 走過最小的前墜和) 得出
    		\end{itemize}
    \end{frame}
    
    \begin{frame}
    		\begin{itemize}
    			\item 枚舉 $a$,找到最大的答案
    			\item 但是 $a \le 10^9$,光是枚舉 $a$ 就超時了,因此我們先對他做離散化,讓 $a < n$
    			\item 若每次尋找最大前墜和的過程都是 $\mathcal{O}(n)$,那麼整體複雜度會來到 $\mathcal{O}(n^2)$
    			\item 觀察一下 $r_a$ 會發現,他其實就是幾個 $+1$ 中間插入很多個 $-1$ 組成的,再加上最好的邊界一定落在 $r_{a,i}=+1$ 的位置,所以我們只需要紀錄那些點是 $+1$ 即可
    			\item 每次計算最大前墜和的複雜度變為 $\mathcal{O}(cnt_a)$
    			\item 最終的複雜度就會是 $\mathcal{O}(\displaystyle\sum{cnt_{a}}) = \mathcal{O}(n)$ 加上離散化的 $\mathcal{O}(n\log n) = \mathcal{O}(n\log n)$
    		\end{itemize}
    \end{frame}

\end{document}